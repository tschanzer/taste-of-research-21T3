\documentclass[handout]{beamer}

\usepackage{hyperref}
\usepackage{soul}
\usepackage{minted}
\usepackage{xcolor}
\definecolor{LightGray}{gray}{0.9}
\setbeamercovered{transparent}
\graphicspath{{figures}}

\title{A Simple Parcel Theory Model of Downdrafts in Atmospheric Convection}
\author{
    Thomas Schanzer
    \texorpdfstring{\\}{}
    \small \texorpdfstring{
        \url{https://github.com/tschanzer/taste-of-research-21T3}}{
        https://github.com/tschanzer/taste-of-research-21T3
    }
    \texorpdfstring{\\ \vspace{5mm}}{}
    Supervisor: Prof. Steven Sherwood}
\institute{UNSW School of Physics}
\date{Monday 22 November 2021}

\begin{document}

\frame{\titlepage}

\begin{frame}
    \frametitle{Aim and Motivation}
    Downdrafts play an important role in the dynamics of the Earth's
    atmosphere and climate.
    \begin{block}{Question}
        Which processes and conditions initiate, and which
        maintain or inhibit, downdrafts?
    \end{block}
    \begin{block}{Motivation}
        \begin{itemize}
            \item Convection parametrisation in global climate models
            \item Forcasting dangerous downbursts
        \end{itemize}
    \end{block}
\end{frame}

\begin{frame}
    \frametitle{Literature}
    \emph{Knupp and Cotton (1985)}
    \footnote{ \tiny
        Knupp, KR \& Cotton, WR 1985, ‘Convective cloud downdraft
        structure: An interpretive survey’, Reviews of geophysics (1985),
        vol. 23, no. 2, pp. 183–215.}
    identify four downdraft types from a review of observational and
    modelling research:
    \begin{itemize}
        \item<1,5> Precipitation-associated, \pause
        \item<2,6> Penetrative, \pause
        \item<3> Cloud-edge, \pause
        \item<4> Overshooting. \pause
    \end{itemize}
    \emph{In this work:} precipitation-associated and penetrative
    downdrafts.
\end{frame}

\begin{frame}
    \frametitle{Background: Parcel Theory}
    \textbf{Parcel:} small air mass with an imaginary, flexible but closed
    boundary.

    \textbf{Key assumptions:}
    \begin{itemize}
        \item Motion is purely vertical and buoyancy is the only force
            involved: \[b = \frac{\rho_E - \rho_P}{\rho_P} g.\]
        \item Raising and lowering the parcel is a reversible adiabatic
            process
    \end{itemize}

    \textbf{Major complication:} the atmosphere contains water!
    \begin{itemize}
        \item Descent is either \emph{dry} adiabatic (no phase changes) or
            \emph{moist} adiabatic (with phase changes)
        \item Phase equilibrium is maintained
        \item Air/vapour mixture is an ideal gas
    \end{itemize}
\end{frame}

\begin{frame}[fragile]
    \frametitle{Methods}
    Original model developed from first principles in Python.
    \begin{itemize}
        \item The \verb|Environment| class interpolates real atmospheric
            temperature and moisture profiles to calculate derived quantities:
\vspace{-3mm}
\begin{minted}[bgcolor=LightGray, fontsize=\footnotesize]{python}
>>> sydney.density(5*units.km)
0.7206758681891053 kilogram/meter^3
\end{minted}
\vspace{-7mm}
        \item Various thermodynamic calculations (approximate and
            exact) are implemented from literature
        \item End goal: calculate parcel temperature $\to$ density $\to$
            buoyancy as functions of height and numerically solve
            \[\frac{d^2 z}{dt^2} = b(z).\]
    \end{itemize}
\end{frame}

\begin{frame}
    \frametitle{Results: Precipitation Enhances Downdrafts}
    \begin{figure}[ht]
        \centering
        \includegraphics[width=0.9\linewidth]%
            {figures/motion_vs_initial_conditions_50RH_1perkm.eps}
    \end{figure}
\end{frame}

\begin{frame}
    \frametitle{Results: Entrainment Inhibits Downdrafts}
    \begin{figure}[ht]
        \centering
        \includegraphics[width=0.9\linewidth]%
            {figures/motion_vs_entr_rate_2gram_50RH.eps}
    \end{figure}
\end{frame}

\begin{frame}
    \frametitle{Results: Atmospheric Dryness Enhances Downdrafts}
    \begin{figure}[ht]
        \centering
        \includegraphics[width=0.9\linewidth]%
            {figures/motion_vs_RH_2gram_1perkm-eps-converted-to.pdf}
    \end{figure}
\end{frame}

\begin{frame}
    \frametitle{Conclusions and Future Work}
    \textbf{Conclusions:}
    \begin{itemize}
        \item Precipitation evaporation increases strength and penetration
        \item Entrainment reduces them
        \item Atmospheric dryness increases them
    \end{itemize}

    \textbf{Application:} supplement basic sounding analysis methods used
    in weather forecasting

    \textbf{Future Work:}
    \begin{itemize}
        \item Consider other forces at play, e.g. drag
        \item Model more advanced dynamics, e.g. entrainment from
            updrafts
    \end{itemize}
\end{frame}

\end{document}