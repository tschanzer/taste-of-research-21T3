\documentclass[12pt,titlepage]{article}
\usepackage[utf8]{inputenc}
\usepackage[a4paper, total={16cm, 22cm}]{geometry}
\usepackage[
	backend=biber,
	bibstyle=authoryear,
	citestyle=authoryear-comp,
	sorting=nyt,
	uniquename=false,
	maxbibnames=99
]{biblatex}
\usepackage{
	fancyhdr,
	amsmath,
	amssymb,
	graphicx,
	siunitx,
	standalone,
	tikz,
	cprotect,
	enumitem,
	hyperref,
	comment,
}


\addbibresource{references.bib}

\setlength{\parskip}{2mm}
\pagestyle{fancy}
\fancyhf{}
\rhead{Thomas Schanzer}
\lhead{}
\lfoot{\leftmark}
\rfoot{Page \thepage}
\setlength{\headheight}{15pt}
\renewcommand{\headrulewidth}{2pt}
\renewcommand{\footrulewidth}{1pt}
\fancypagestyle{first}{
	\fancyhead{}
	\fancyfoot{}
	\renewcommand{\headrulewidth}{0pt}
	\renewcommand{\footrulewidth}{0pt}
}

\newcommand{\dcape}{\mathrm{DCAPE}}
\newcommand{\dcin}{\mathrm{DCIN}}

\begin{document}

\hypersetup{pageanchor=false}
\begin{titlepage}
    \begin{center}
        ~

        \vspace{3cm}
        \Huge
        \textbf{%
        	A Simple Parcel Theory Model of Downdrafts in Atmospheric
        	Convection}
        
        \vspace{0.75cm}
        \Large
        \textbf{Thomas D. Schanzer}
            
        \vfill
            
        \includegraphics[width=0.25\textwidth]{figures/unsw}

        \vspace{1cm}

        \large    
        Taste of Research 2021

        Supervisor: Prof. Steven Sherwood

        \vspace{1cm}
            
        \large
        School of Physics\\
        Faculty of Science\\
        University of New South Wales\\
        Sydney, Australia\\
    \end{center}
\end{titlepage}

\hypersetup{pageanchor=true}
\addtocounter{page}{1}
\begin{center}
	\large
	\textbf{Abstract}
\end{center}
Downdrafts are known to play an important role in the dynamics of the
Earth's atmosphere and climate, but current quantitative understanding
of the mechanisms that generate, maintain and inhibit them is lacking.
A simple original model, based on the parcel theory that considers
only buoyant forces, is devised to investigate the factors that
determine the speed with which a negatively buoyant air mass descends,
and the distance it is able to penetrate in the atmosphere. It is found
that the evaporation of precipitation is a strong initiating mechanism,
that the entrainment of environmental air into a downdraft hinders
its descent, and that dryness in the atmosphere above the boundary layer
accelerates it, in agreement with the observations and
more sophisticated models documented in existing literature.


\begin{center}
	\large
	\textbf{Acknowledgements}
\end{center}
The author is most grateful to his supervisor, Prof. Steven Sherwood, for
his patient guidance and mentorship during the ten-week research project
whose outcomes are documented in this report. The author also thanks
the members of Prof. Sherwood's research group at the UNSW Climate
Change Research Centre for some useful suggestions and feedback.

The author would also like to thank the UNSW School of Physics for
offering the Taste of Research program under which this project was
conducted, and in particular the program coordinator, A. Prof. Sarah
Martell. The program has proved to be a very valuable and enjoyable
learning opportunity.

\begin{center}
	\large
	\textbf{Data availability statement}
\end{center}
All Python scripts and notebooks used to produce the results in this
report are publicly available at
\url{https://github.com/tschanzer/taste-of-research-21T3}.

\begin{center}
	\large
	\textbf{Notation for thermodynamic variables and constants}
\end{center}
\begin{description}[align=right,labelwidth=2cm,itemsep=0mm]
	\item[$p$] Total pressure
	\item[$z$] Height
	\item[$T$] Absolute temperature
	\item[$T_W$] Wet bulb temperature
	\item[$\rho$] Density
	\item[$q$] Specific humidity
	\item[$q^*(p,T)$] Saturation specific humidity
	\item[$\mathrm{RH}$] Relative humidity
	\item[$\theta_e(p,T,q)$] Equivalent potential temperature
	\item[$l$] Ratio of liquid water mass to total
	\item[$\lambda$] Entrainment rate
	\item[$L_v$] Latent heat of vapourisation of water
	\item[$c_p$] Specfic heat at constant pressure of dry air
	\item[$R$] Specific gas constant of dry air
	\item[$g$] Acceleration due to gravity
	\item[$(\cdot)_E$] Environmental quantity
	\item[$(\cdot)_P$] Parcel quantity
\end{description}

\tableofcontents

\clearpage
\section{Introduction and theory}
Along with updrafts, downdrafts---downward-moving masses of air---are
important features in the dynamics of the Earth's atmosphere;
they transport mass, momentum, heat and moisture vertically
and also generate and maintain storms \parencite{knupp_cotton_1985}.

Indeed, one of the main objectives of present-day research into
downdraft dynamics is to improve the predictions of global climate
models \parencite{thayer-calder_2013}, whose output informs the
understanding of the larger-scale
dynamics, including the pressing issue of anthropogenic climate
change. Specifically, the high computational cost of running a global
climate model over the necessarily large spatial domain and prediction
timescales constrains their maximum resolution, which is still too
coarse to describe convection. The models therefore employ schemes
known as \emph{parametrisations} which estimate the effect of
convection on the state of the model using the information available
at each time step; an accurate estimation requires a strong
understanding of the factors that govern convection.

On a smaller scale, strong downdrafts that reach the Earth's surface
(\emph{downbursts}) are known to cause significant damage to
man-made structures and create hazardous, or even deadly, conditions
for aircraft \parencite{thayer-calder_2013}. Another aim of downdraft
research is therefore to understand the mechanisms that generate
such extreme events and improve the ability to predict them in advance.

Considering these motivations, the goal of this work is to gain
insight into which processes and conditions initiate, and which
maintain or inhibit, downdrafts. The approach will be to construct
a significantly simplified model of a downdraft using \emph{parcel
theory}.

An air parcel is a mass of air with an imaginary flexible (but usually
closed) boundary; under the usual assumptions, its exact size and
shape are irrelevant. The only force assumed to act on the parcel is
the net buoyant force (per unit mass), given in accordance with
Archimedes' principle by
\begin{equation}
	b = \frac{\rho_E - \rho_P}{\rho_P} g.
	\label{eqn:buoyancy}
\end{equation}
If the parcel is lowered in the atmosphere to a location with a higher
pressure, the work done to compress it and any heat exchanged will
manifest as a change in its internal energy in accordance with the
first law of thermodynamics. The second key assumption of parcel theory
is that this process is adiabatic; this is valid due to the low
thermal conductivity of air.

The potential presence of water in gas, liquid and solid phases in the
parcel is a major complication; under the assumption that the parcel
remains in phase equilibrium (i.e., changes are slow enough for
excess liquid to evaporate if the vapour pressure is below the
saturation value), there are two modes of adiabatic descent the parcel
may undergo. If no liquid is present, the descent is \emph{dry
adiabatic} and the rate of work on the parcel causes it to warm at
an approximate rate of $\SI{9.8}{\kelvin \per\kilo\meter}$.
If liquid is present, the descent is \emph{moist adiabatic}:
progressive warming of the parcel raises its saturation vapour pressure,
allowing the liquid to progressively evaporate during descent,
with the necessary transfer of latent heat from the air to the water
creating an opposing cooling effect.

Moist adiabatic descent is commonly assumed to be either
\emph{pseudoadiabatic}, in which case liquid water does not contribute
to the heat capacity of the parcel (as if it precipitates from the
parcel immediately upon condensation), or \emph{reversible}, in
which case the liquid does contribute to the heat capacity.
A reversibly descending parcel warms at a slightly slower rate than a
pseudoadiabatically descending one due to its larger heat capacity
\textcite{saunders_1957}.
Both modes were investigated, but reversible descent was ultimately
chosen as it is the more realistic case for a parcel known to retain
liquid water.

If the pressure and temperature of the parcel are thus known at any
point in its descent, its density may be calculated using the ideal
gas law,
\begin{equation}
	\rho = \frac{p}{RT_v}, \label{eqn:density}
\end{equation}
where $T_v$ is the \emph{virtual temperature} that contains a small
correction to account for the different density of water vapour.
If an mass $l$ of liquid water, per unit total parcel mass, is also
present, it is easily shown that (assuming the liquid occupies
negligible volume) the corrected parcel density is
\begin{equation}
	\rho = \frac{p}{RT_v (1 - l)}.
\end{equation}
Knowledge of the parcel and environmental densities enable calculation
of the buoyant force per unit mass on the parcel using
(\ref{eqn:buoyancy}), and its resulting displacement and velocity may be
obtained by (numerically) solving the ODE
\begin{equation}
	\frac{\mathrm{d}^2 z}{\mathrm{d}t^2} = b(z).
	\label{eqn:ode}
\end{equation}


\section{Literature review}
% OUTLINE
% 	- Knupp/Cotton review paper, downdraft types
% 	- Thayer-Calder thesis
% 	- Market 2017 on DCAPE, DCIN correlation
\begin{comment}
Possible items for discussion:
\begin{itemize}
	\item \textcite{knupp_cotton_1985}: the downdraft types and their
		typical characteristics
	\item \textcite{thayer-calder_2013}: last chapter on the Lagrangian
		view of downdrafts
	\item \textcite{market_2017}: correlation between DCAPE and DCIN and
		downdraft strength
	\item \textcite{sumrall_2020}: DCAPE, DCIN and severe surface winds
	\item \textcite{davies-jones_2008}: pseudoadiabatic wet bulb
		temperature approximations
	\item \textcite{bolton_1980}: equivalent potential temperature
		and saturation vapour pressure approximations
	\item \textcite{saunders_1957}: reversible moist adiabatic
		temperature approximation
\end{itemize}
\end{comment}

% \subsection{Downdraft dynamics}
In their review of existing literature,
\textcite{knupp_cotton_1985} identify four downdraft types based
on the mechanisms that generate or maintain them, which
informed the methods of this investigation: penetrative, cloud-edge,
overshooting and precipitation-associated.
The penetrative type arises when subsaturated environmental air
is \emph{entrained} (mixed) into saturated cloudy air, allowing
evaporation of the excess liquid that creates negative buoyancy.
The cloud-edge type, they note, is less understood and may result
from evaporative cooling at the edges of clouds.
An overshooting downdraft may be generated when the inertia of
an updraft causes it to rise beyond its level of neutral buoyancy
and subsequently sink.
The precipitation-associated downdraft is generated by the evaporation
of precipitation into subsaturated air beneath a cloud, with the cooling
creating negative buoyancy. They note that $\SI{20}{\meter \per\second}$
is a typical upper limit on downdraft velocity in all cases. The model
presented in
this work most closely describes the precipitation-assiciated and
penetrative types.

\textcite{knupp_cotton_1985} find that downdrafts often become
subsaturated during descent, a conclusion reproduced by
\textcite{thayer-calder_2013} in a Lagrangian (parcel-tracking)
study of the output of a cloud-resolving model. The latter also found
that downdrafts often descend past their neutral buoyancy levels.
These findings are clearly reproduced in this work.

\textcite{market_2017} use case studies to investigate the dependence
of downdraft
penetration depth on quantities known as \emph{downdraft convective
available potential energy} (DCAPE) and \emph{downdraft convective
inhibition} (DCIN). The former is a measure of the maximum kinetic
energy per unit mass a parcel in a given environment may gain,
and the latter is a measure of the work necessary to bring the parcel
to the ground once it passes its neutral buoyancy level, against
the upward buoyant force. Both are calculated as integrals of the
buoyant force with respect to height between chosen initial and final
levels (bracketing
regions of positive buoyancy for DCAPE and negative buoyancy for DCIN).
They come to the conclusion that the smaller
the DCIN relative to the DCAPE, the more likely downdrafts are to
penetrate stable parts of the atmosphere and reach the ground.
\textcite{sumrall_2020} conducts further case study investigations,
with findings supporting a hypothesis that small ratios of
$|\dcin/\dcape|$ are correlated with stronger and deeper downdraft
activity and surface winds, and vice versa for large ratios.
This work also reproduces the above findings.

% \subsection{Thermodynamic calculations}


\section{Methods} \label{section:methods}
% OUTLINE
% 	- Additional assumptions
% 	- Code structure
\subsection{General approach and code structure}
All calculations were performed in Python 3.8.5. Other than
basic thermodynamic calculations (such as saturation specific humidity
from pressure and temperature, and density according to
(\ref{eqn:density})) which were imported from the open-source
\verb|MetPy| package \parencite{metpy}, the model was built from
scratch and from first principles. The model is divided between
three scripts, whose functions are described below.

\verb|environment.py| defines the \verb|Environment| class, which
provides utilities for finding the various environmental variables
at any height, allowing the model to be applicable to any real or
idealised atmospheric \emph{sounding} (vertical profile of environmental
variables). The class is instantiated by supplying discrete vertical
profiles of pressure, temperature and dewpoint, and the resulting
instance offers methods that interpolate the data and calculate
derived quantities at any height. For example, the density at
$\SI{5}{\kilo\meter}$ in an \verb|Environment| named \verb|sydney|,
initialised with a measured sounding over Sydney, is given by
\verb|sydney.density(5*units.kilometer)|.

\verb|thermo.py| provides various thermodynamic calculations that
are not available in \verb|MetPy|. The following important functions are
implementations of existing published methods:
\begin{itemize}
	\item \emph{Equivalent potential temperature} (a variable with
		dimensions of temperature that is conserved during moist
		adiabatic descent) as a function of pressure,
		temperature and specific humidity, following the empirical
		formula presented by \textcite{bolton_1980} in his Equation
		(39),
	\item DCAPE and DCIN for a given atmospheric sounding, following
		the definitions of \textcite{market_2017},
	\item The exact analytical solution for the \emph{lifting
		condensation level} (LCL), the pressure level to which a parcel must
		be lifted in order to cool it to the point of saturation and
		the temperature at this point, adapting the implementation
		of \textcite{romps_2017},
	\item The \emph{wet bulb temperature} (the temperature to which
		a parcel is cooled by evaporation of water to the point of
		saturation, at constant pressure), using Normand's rule.
		Normand's rule states that if a parcel is lifted dry
		adiabaically to its LCL (which we find using the method of
		\textcite{romps_2017}) and lowered moist adiabatically back
		to the initial level, it attains its wet bulb temperature.
	\item An approximate, but faster, wet bulb temperature calculation
		implementing the method of \textcite{davies-jones_2008}, and
	\item The parcel temperature resulting from reversible adiabatic
		ascent or descent, implementing a numerical solution of
		Equation (3) presented by \textcite{saunders_1957}.
\end{itemize}
Additional original functions written for the model are described
in the subsections that follow.

\verb|entraining_parcel.py| defines the \verb|EntrainingParcel| class,
which assembles the functions of the other two scripts to create
the final model. It is instantiated by supplying an \verb|Environment|
instance and offers methods for calculating parcel displacement
and velocity as functions of time, given some initial conditions,
on that atmospheric sounding. These are described in detail in
the following subsections.


\subsection{Temperature of an entraining, descending parcel}
The theory of the model developed in this work is a generalisation
of the standard parcel theory in that it accounts for a parcel that
continually entrain air from its environment, exchanging heat
and liquid and/or gaseous water. Following the generally accepted
pattern common to the approaches documented by
\textcite{knupp_cotton_1985}, the cumulative mass exchanged by
entrainment is assumed to be a linear function of distance descended,
with the constant of proportionality $\lambda$ termed the
\emph{entrainment rate}. The dimension of $\lambda$ is inverse length;
for a parcel of mass $m$, the total mass exchanged after it descends
a distance $\Delta z$ is $m \lambda \Delta z$. One of the main
contributions of this work is to develop a method for calculating
the temperature of a parcel that simultaneously descends and entrains
in this manner. The general approach is to divide the descent into
small vertical steps, each one consisting of a discrete entrainment
process and a short adiabatic descent.


\subsubsection*{Mixing and phase equilibration}
The first step towards finding the temperature of an entraining and
descending parcel is to find the state that results from a discrete
entrainment step, without descent. The procedure is depicted in
Figure \ref{fig:mix_equilibrate_flowchart} and is based on the
conservation of total enthalpy and water mass, and the requirement
that the parcel return to phase equilibrium after its exchange with
the environment (by either condensing or evaporating water). In the
cases where evaporation or condensation is necessary, we recognise
that the resulting temperature is the parcel's wet bulb temperature.

\begin{figure}[ht]
	\centering
	\includegraphics[width=0.9\linewidth]%
		{figures/mix_equilibrate_flowchart.pdf}
	\cprotect\caption{%
		Flowchart for the mixing and phase equilibration calculation
		(functions $\verb|mix|$ and $\verb|equilibrate|$)
		performed at each downward step for the entraining downdraft.
	}
	\label{fig:mix_equilibrate_flowchart}
\end{figure}


\subsubsection*{Dry and/or reversible moist adiabatic descent}
After the parcel has mixed with the environment and returned to phase
equilibrium, it descends adiabatically. The central complication in
this step is that the descent may be dry (if no liquid water is
present in the parcel), moist (if liquid water is present), or
part moist followed by part dry (if the small amount of liquid water
present fully evaporates midway during descent). If this third case
is necessary, the final temperature may be computed by recognising
that the parcel conserves its equivalent potential temperature,
and therefore finding the unique final temperature that satisfies
this requirement using numerical root-finding. Figure
\ref{fig:descend_flowchart} shows all the steps involved.

\begin{figure}[ht]
	\centering
	\includegraphics[width=0.75\linewidth]%
		{figures/descend_flowchart.pdf}
	\cprotect\caption{%
		Flowchart for the descent calculation ($\verb|descend|$)
		performed at each downward step for the entraining downdraft.
	}
	\label{fig:descend_flowchart}
\end{figure}


\subsubsection*{Finding temperature as a function of height}
The previous two procedures, performed one after the other, give
the state of the parcel after a small discrete downward step. A
calculation that is valid over larger distances must divide the
vertical interval into small steps ($\SI{50}{\meter}$ was found
to give sufficient convergence by trial and error) and repeat the
discrete method at each one until the desired final height is reached.
This process is depicted in Figure \ref{fig:profile_flowchart}.

\begin{figure}[ht]
	\centering
	\includegraphics[width=0.6\linewidth]%
		{figures/profile_flowchart.pdf}
	\cprotect\caption{%
		Flowchart for the calculation of parcel temperature as a
		function of height ($\verb|EntrainingParcel.profile|$),
		assembling the routines
		shown in Figures \ref{fig:mix_equilibrate_flowchart} and
		\ref{fig:descend_flowchart}.
	}
	\label{fig:profile_flowchart}
\end{figure}

\subsection{Density, buoyancy and motion of a descending parcel}
Now, knowing the temperature of the parcel as a function of
height/pressure, it is a relatively simple matter to find its
density using (\ref{eqn:density}) and the buoyant force per unit mass
acting on it using (\ref{eqn:buoyancy}). We then numerically solve
(\ref{eqn:ode}), expressed as the first-order system
\begin{equation*}
	\frac{\mathrm{d}}{\mathrm{d}t}
	\begin{pmatrix} z \\ w\end{pmatrix}
	= \begin{pmatrix} w \\ b(z) \end{pmatrix}.
\end{equation*}


\clearpage
\section{Results}
% OUTLINE
% 	Initial conditions (what initiates a downdraft)
% 	- Vary amount of initial evaporation, with and without liquid water
%
% 	Environmental factors (what maintains a downdraft)
% 	- Vary entrainment rate for fixed initial conditions and sounding
% 	- Vary sounding RH for fixed entrainment rate and initial conditions
% 		- DCAPE/DCIN correlation?

The model developed in Section \ref{section:methods} allows for the
variation of numerous parameters relevant to downdrafts. One may
specify arbitrary initial conditions for height, temperature,
specific humidity and liquid water mass ratio, choose the entrainment
rate, and use any atmospheric sounding to generate the environmental
variables.

\subsection{Downdraft initiation and initial conditions}%
\label{section:results_initial_conditions}

We first envision a downdraft that is initiated by precipitation
falling into subsaturated environmental air (the
precipitation-associated type identified by \textcite{knupp_cotton_1985}),
and investigate the effect of the amount of water evaporated (and the
consequent cooling) on downdraft strength and penetration.

An idealised atmospheric sounding is used, with a dry adiabatic
temperature profile in a boundary layer near the surface, a capping
\emph{inversion} (region of increasing temperature with height) above
it, and a moist adiabatic temperature profile in the remaining upper
atmosphere, whose relative humidity is constant at 50\%. The temperature
and dew point profiles in the sounding are depicted in Figure
\ref{fig:initial_conditions_skewt}. The initial
height is set to $\SI{5}{\kilo\meter}$, the entrainment rate to
$\SI{1}{\per\kilo\meter}$ and we test both the case
where the initial precipitation does not saturate the parel, and the
case where it does, depositing a variable amount of liquid water
in addition.

\begin{figure}[ht]
	\centering
	\includegraphics[width=0.4\linewidth]%
		{figures/20211110_varying_entrainment_figures/skewt.eps}
	\caption{
		Skew $T$-$\log p$ plot of the idealised atmospheric sounding used in
		Section \ref{section:results_initial_conditions}.}
	\label{fig:initial_conditions_skewt}
\end{figure}

Figure \ref{fig:initial_conditions_motion} shows the height and
velocity of the parcels with different sets of initial conditions,
and the minimum height and maximum velocity reached as a function of
the cumulative amount of water initially added by precipitation.
It is clear that both downdraft strength (maximum velocity) and
penetration (initial minus final height) are enhanced by precipitation.
This is expected; greater initial evaporation of precipitation reduces
the parcel's temperature further, causing it to be more negatively
buoyant and to descend further before coming into equilibrium with the
environment. The addition of liquid water by precipitation also
stregthens these effects, since it contributes to the parcel's weight
and allows it to descend moist adiabatically; under this regime, the
increase in temperature per unit distance descended is smaller and
the parcel again must descend further, and has more time to gain
velocity, before coming into equilibrium with the environment.

\begin{figure}[ht]
	\centering
	\includegraphics[width=0.9\linewidth]%
		{figures/20211110_varying_initial_conditions_figures/%
		motion_vs_initial_conditions_50RH_1perkm.eps}
	\caption{
		Properties of a downdraft parcel originating at height
		$\SI{5}{\kilo\meter}$ in an idealised atmospheric sounding with
		50\% relative humidity in the upper atmosphere and a fixed
		entrainment rate of $\SI{1}{\per\kilo\meter}$. Top row:
		height (left) and velocity (right) as functions of time, for
		selected initial conditions. Bottom row: minimum height reached
		(left) and maximum downward velocity (right) as functions of
		the total amount of water initially added to the parcel
		(specific humidity change due to evaporation $\Delta q$ plus
		additional liquid water per unit parcel mass $\Delta l$).
	}
	\label{fig:initial_conditions_motion}
\end{figure}


\clearpage
\subsection{The impact of entrainment} \label{section:results_entrainment}
We now investigate the effect of
entrainment rate on downdraft strength and penetration. The scenario
is similar to the one described in Section
\ref{section:results_initial_conditions}. The initial conditions are
fixed: the parcel is initially saturated by precipitation, which
deposits $\SI{2}{\gram \per\kilo\gram}$ liquid water in it. The
atmospheric sounding used is the same. The entrainment rate is varied
between $\SI{0}{\per\kilo\meter}$ and $\SI{2}{\per\kilo\meter}$, which
covers typical measured values \parencite{lu_2016}.

Figure \ref{fig:entrainment_motion} depicts the same dependent variables
as Figure \ref{fig:initial_conditions_motion}, with the key independent
variable now being entrainment rate. We observe that entrainment
clearly reduces the strength and penetration of the downdraft; this
is attributed to the fact that more vigorous mixing causes the parcel
to come into equilibrium with its environment sooner, before it is
able to descend as far or gain as much velocity as a parcel experiencing
less mixing.{}

\begin{figure}[ht]
	\centering
	\includegraphics[width=0.9\linewidth]%
		{figures/20211110_varying_entrainment_figures/%
		motion_vs_entr_rate_2gram_50RH.eps}
	\caption{
		Properties of a downdraft parcel originating at height
		$\SI{5}{\kilo\meter}$ in an idealised atmospheric sounding with
		50\% relative humidity in the upper atmosphere. The initial
		conditions are fixed: an environmental parcel is brought to
		saturation by evaporation of liquid water, and
		$\SI{2}{\gram \per\kilo\gram}$ liquid water is additionally
		suspended in the parcel.
		Top row: height and velocity over time for selected entrainment
		rates.
		Bottom row: minimum height reached and maximum velocity as
		functions of entrainment rate.
	}
	\label{fig:entrainment_motion}
\end{figure}


\clearpage
\subsection{The impact of environmental humidity}%
\label{section:results_humidity}
In the third set of calculations, the environment in which
the parcel's motion takes place is varied by changing the relative
humidity in the upper part of the idealised sounding described in
Section \ref{section:results_initial_conditions}. The resulting
temperature and dew point profiles are shown in Figure
\ref{fig:env_humidity_skewt}. The initial conditions are similar
to those of Section \ref{section:results_entrainment}: the parcel is
initially saturated by precipitation, which
deposits $\SI{2}{\gram \per\kilo\gram}$ liquid water in it. However,
the varying environmental relative humidity causes the amount of
precipitation that may evaporate, and the resulting initial temperature,
to vary; in a sounding with lower relative humidity, the parcel
experiences more initial evaporation and cooling.

\begin{figure}[ht]
	\centering
	\includegraphics[width=0.5\linewidth]%
		{figures/20211110_varying_env_humidity_figures/skewt.eps}
	\caption{
		Skew $T$-$\log p$ plot of some selected idealised atmospheric
		soundings used in Section \ref{section:results_humidity}.
		The dashed lines on the left are the dewpoint profiles for
		the different soundings, and the solid blue line on the right
		is the common temperature profile.
	}
	\label{fig:env_humidity_skewt}
\end{figure}

Figure \ref{fig:env_humidity_motion} now depicts the same dependent variables
as Figures \ref{fig:initial_conditions_motion} and
\ref{fig:entrainment_motion}, with the key independent
variable being environmental relative humidity. We observe that
the downdraft is stronger and penetrates further in the drier
environments due to the greater degree of initial precipitation-induced
cooling that is possible, as previously described.

\begin{figure}[ht]
	\centering
	\includegraphics[width=0.9\linewidth]%
		{figures/20211110_varying_env_humidity_figures/%
		motion_vs_RH_2gram_1perkm.eps}
	\caption{
		Properties of a downdraft parcel originating at height
		$\SI{5}{\kilo\meter}$ in idealised atmospheric soundings whose
		upper atmosphere relative humidities vary between 30\% and 90\%.
		The initial conditions are generated by bringing an environmental
		parcel to saturation by evaporation of liquid water (note that
		the resulting temperatures differ since more humid environmental
		parcels are closer to their wet bulb temperatures), and
		$\SI{2}{\gram \per\kilo\gram}$ liquid water is additionally
		suspended in the parcel.
		Top row: height and velocity of the parcel over time for
		selected soundings.
		Bottom row: minimum height reached and maximum downward
		velocity as functions of relative humidity in the upper
		atmosphere.
	}
	\label{fig:env_humidity_motion}
\end{figure}

Furthermore, the DCAPE and DCIN were computed for each sounding,
allowing the plot of downdraft strength and penetration as functions
of these variables shown in Figure \ref{fig:env_humidity_dcape_dcin}.
The results are in strong agreement with the findings of
\textcite{market_2017} and \textcite{sumrall_2020}: the greater the
potential energy available and the smaller the inhibition in its way,
the further and faster the parcel descends. We also note that,
as postulated by \textcite{sumrall_2020}, smaller ratios of
$|\dcape/\dcin|$ are strongly correlated with increased downdraft
strength and penetration.

\begin{figure}[ht]
	\centering
	\includegraphics[width=0.95\linewidth]%
		{figures/20211110_varying_env_humidity_figures/%
		strength_vs_dcape_dcin_2gram_1perkm.pdf}
	\caption{
		Plots of the minimum height (top row) and maximum downward
		velocity (bottom row) reached by the parcel of Figure
		\ref{fig:env_humidity_motion} as functions of the
		downdraft convective available potential energy ($\dcape$,
		left column), downdraft convective inhibition ($\dcin$,
		centre column) and the ratio $|\dcin/\dcape|$ (right column).
	}
	\label{fig:env_humidity_dcape_dcin}
\end{figure}


\clearpage
\section{Conclusions}
The construction and testing of a simplified downdraft model, using
parcel theory with modifications for entrainment, was highly successful
in reproducing the behaviour documented in literature.

There is strong
evidence that evaporation of precipitation initiates downdrafts,
consistent with the precipitation-associated downdraft type
identified by \textcite{knupp_cotton_1985}, with the amount of
evaporation and liquid water deposited correlating positively with
downdraft strength (measured by maximum downward velocity) and
penetration (measured by the minimum height reached). We may draw
the logical conclusion that downdrafts are stronger, more prevalent
and potentially more damaging during storms with heavy precipitation.

It is also observed that the entrainment of environmental air into
a downdraft impedes its motion. If it is true that the entrainment
rate is inversely proportional to cloud radius
\parencite{knupp_cotton_1985}, we may conclude that larger clouds,
with smaller entrainment rates, should typically produce stronger,
potentially more damaging downdrafts.

It is finally seen that dryness in the atmosphere above the boundary
layer allows and maintains stronger downdrafts. Larger values of
DCAPE, and smaller values of DCIN and $|\dcin/\dcape|$, are associated
with stronger and deeper penetrating downdrafts, in excellent agreement
with the findings of \textcite{market_2017} and \textcite{sumrall_2020}.

While the simplicity of the model, especially the assumption that
no forces other than buoyancy act on parcels, limit the validity of
the numerical values generated, it is nevertheless able to
unambiguously show the important features of downdraft dynamics in
relative terms. Given its simplicity and ease of use, one proposed use
for the model is as a supplement to the basic techniques of sounding
analysis (such as DCAPE and DCIN calculations) commonly used in
weather forecasting. It could be used to estimate the potential for
damaging downdrafts based on real atmospheric sounding data within minutes
of the data being collected.

Further work may seek to determine which forces other than buoyancy,
such as drag, can be incorporated into the model to improve the
accuracy and validity of its numerical results without excessive
complication and redesigning of the underlying code. A more advanced
model might also seek to account for more advanced downdraft dynamics,
such as entrainment from an adjacent updraft with momentum transfer
in addition to heat and water.


\clearpage
\printbibliography

\end{document}