\documentclass[12pt,titlepage]{article}
\usepackage[utf8]{inputenc}
\usepackage[a4paper, total={16cm, 22cm}]{geometry}
\usepackage[
	backend=biber,
	bibstyle=authoryear,
	citestyle=authoryear-comp,
	sorting=nyt,
	uniquename=false,
	maxbibnames=99
]{biblatex}
\usepackage{
	fancyhdr,
	amsmath,
	amssymb,
	graphicx,
	siunitx,
}


% \addbibresource{references.bib}

\setlength{\parskip}{1em}
\pagestyle{fancy}
\fancyhf{}
\rhead{Thomas Schanzer}
\lhead{}
\lfoot{\leftmark}
\rfoot{Page \thepage}
\setlength{\headheight}{15pt}
\renewcommand{\headrulewidth}{2pt}
\renewcommand{\footrulewidth}{1pt}
\fancypagestyle{first}{
	\fancyhead{}
	\fancyfoot{}
	\renewcommand{\headrulewidth}{0pt}
	\renewcommand{\footrulewidth}{0pt}
}

\newcommand{\dcape}{\mathrm{DCAPE}}
\newcommand{\dcin}{\mathrm{DCIN}}

\begin{document}

% \includepdf[pages=-]{coversheet.pdf}

\begin{titlepage}
    \begin{center}
        ~

        \vspace{3cm}
        \Huge
        \textbf{%
        	A Simple Parcel Theory Model of Downdrafts in Convective Clouds}
        
        \vspace{0.75cm}
        \Large
        \textbf{Thomas D. Schanzer}
            
        \vfill
            
        \includegraphics[width=0.25\textwidth]{figures/unsw}

        \vspace{1cm}

        \large    
        Taste of Research 2021

        Supervisor: Prof. Steven Sherwood

        \vspace{1cm}
            
        \large
        School of Physics\\
        Faculty of Science\\
        University of New South Wales\\
        Sydney, Australia\\
    \end{center}
\end{titlepage}

\begin{center}
	\large
	\textbf{Abstract}
\end{center}

Abstract

\begin{center}
	\large
	\textbf{Acknowledgements}
\end{center}

The author is most grateful to his supervisor, Prof. Steven Sherwood, for
his patient guidance and mentorship during the ten-week research project
whose outcomes are documented in this report. The author also thanks
the members of Prof. Sherwood's research group at the UNSW Climate
Change Research Centre for some useful suggestions and feedback.

The author would also like to thank the UNSW School of Physics for
offering the Taste of Research program under which this project was
conducted, and in particular the program coordinator, A. Prof. Sarah
Martell. The program has proved to be a very valuable and enjoyable
learning opportunity.

\tableofcontents

\clearpage
\section{Introduction and theory}
% OUTLINE
% 	- Aim and motivation
% 	- Introduction to parcel theory and inherent assumptions

\section{Literature review}
% OUTLINE
% 	- Knupp/Cotton review paper, downdraft types
% 	- Thayer-Calder thesis
% 	- Market 2017 on DCAPE, DCIN correlation

\section{Methods}
% OUTLINE
% 	- Additional assumptions
% 	- Code structure
% 	-

\section{Results}
% OUTLINE
% 	Initial conditions (what initiates a downdraft)
% 	- Vary amount of initial evaporation, with and without liquid water
%
% 	Environmental factors (what maintains a downdraft)
% 	- Vary entrainment rate for fixed initial conditions and sounding
% 	- Vary sounding RH for fixed entrainment rate and initial conditions
% 		- DCAPE/DCIN correlation?

\subsection{Downdraft initiation and initial conditions}%
\label{section:results_initial_conditions}

\begin{figure}[ht]
	\centering
	\includegraphics[width=0.4\linewidth]%
		{figures/20211110_varying_entrainment_figures/skewt.eps}
	\caption{
		Skew $T$-$\log p$ plot of the idealised atmospheric sounding used in
		Section \ref{section:results_initial_conditions}.}
	\label{fig:initial_conditions_skewt}
\end{figure}

\begin{figure}[ht]
	\centering
	\includegraphics[width=0.9\linewidth]%
		{figures/20211110_varying_initial_conditions_figures/%
		motion_vs_initial_conditions_50RH_1perkm.eps}
	\caption{
		Properties of a downdraft parcel originating at height
		$\SI{5}{\kilo\meter}$ in an idealised atmospheric sounding with
		50\% relative humidity in the upper atmosphere and a fixed
		entrainment rate of $\SI{1}{\per\kilo\meter}$. Top row:
		height (left) and velocity (right) as functions of time, for
		selected initial conditions. Bottom row: minimum height reached
		(left) and maximum downward velocity (right) as functions of
		the total amount of water initially added to the parcel
		(specific humidity change due to evaporation $\Delta q$ plus
		additional liquid water per unit parcel mass $\Delta l$).
	}
	\label{fig:initial_conditions_motion}
\end{figure}


\subsection{The impact of entrainment}

\begin{figure}[ht]
	\centering
	\includegraphics[width=0.9\linewidth]%
		{figures/20211110_varying_entrainment_figures/%
		motion_vs_entr_rate_2gram_50RH.eps}
	\caption{
		Properties of a downdraft parcel originating at height
		$\SI{5}{\kilo\meter}$ in an idealised atmospheric sounding with
		50\% relative humidity in the upper atmosphere. The initial
		conditions are fixed: an environmental parcel is brought to
		saturation by evaporation of liquid water, and
		$\SI{2}{\gram \per\kilo\gram}$ liquid water is additionally
		suspended in the parcel.
		Top row: height and velocity over time for selected entrainment
		rates.
		Bottom row: minimum height reached and maximum velocity as
		functions of entrainment rate.
	}
	\label{fig:entrainment_motion}
\end{figure}



\subsection{The impact of environmental humidity}%
\label{section:results_humidity}

\begin{figure}[ht]
	\centering
	\includegraphics[width=0.5\linewidth]%
		{figures/20211110_varying_env_humidity_figures/skewt.eps}
	\caption{
		Skew $T$-$\log p$ plot of some selected idealised atmospheric
		soundings used in Section \ref{section:results_humidity}.
		The dashed lines on the left are the dewpoint profiles for
		the different soundings, and the solid blue line on the right
		is the common temperature profile.
	}
	\label{fig:env_humidity_skew1}
\end{figure}

\begin{figure}[ht]
	\centering
	\includegraphics[width=0.9\linewidth]%
		{figures/20211110_varying_env_humidity_figures/%
		motion_vs_RH_2gram_1perkm.eps}
	\caption{
		Properties of a downdraft parcel originating at height
		$\SI{5}{\kilo\meter}$ in idealised atmospheric soundings whose
		upper atmosphere relative humidities vary between 30\% and 90\%.
		The initial conditions are generated by bringing an environmental
		parcel to saturation by evaporation of liquid water (note that
		the resulting temperatures differ since more humid environmental
		parcels are closer to their wet bulb temperatures), and
		$\SI{2}{\gram \per\kilo\gram}$ liquid water is additionally
		suspended in the parcel.
		Top row: height and velocity of the parcel over time for
		selected soundings.
		Bottom row: minimum height reached and maximum downward
		velocity as functions of relative humidity in the upper
		atmosphere.
	}
	\label{fig:env_humidity_motion}
\end{figure}

\begin{figure}[ht]
	\centering
	\includegraphics[width=0.95\linewidth]%
		{figures/20211110_varying_env_humidity_figures/%
		strength_vs_dcape_dcin_2gram_1perkm.pdf}
	\caption{
		Plots of the minimum height (top row) and maximum downward
		velocity (bottom row) reached by the parcel of Figure
		\ref{fig:env_humidity_motion} as functions of the
		downdraft convective available potential energy ($\dcape$,
		left column), downdraft convective inhibition ($\dcin$,
		centre column) and the ratio $|\dcin/\dcape|$ (right column).
	}
	\label{fig:env_humidity_dcape_dcin}
\end{figure}


\section{Conclusions}

% \clearpage
% \printbibliography

\end{document}